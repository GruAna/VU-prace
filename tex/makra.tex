%% --- zde jsou zavedeny některé "konstanty" - některé musíte změnit! --- %%
\newcommand{\cvut}{České vysoké učení technické v~Praze}
\newcommand{\fjfi}{Fakulta jaderná a fyzikálně inženýrská}
\newcommand{\ksi}{Katedra softwarového inženýrství}
\newcommand{\km}{Katedra matematiky}
% \newcommand{\program}{Aplikace přírodních věd} % změňte, pokud máte jiný stud. program
\newcommand{\obor}{Aplikace informatiky v přírodních vědách} % změňte, pokud máte kurzívujiný obor

\newcommand{\druh}{Výzkumný úkol} % nebo "Diplomová práce"
\newcommand{\woman}{a} % pokud jste ŽENA, ZMĚŇTE na: ...{\woman}{a} (je to do Prohlášení)

\newcommand{\logoCVUT}{\includegraphics{symbol_cvut_konturova_verze_cb.pdf}} % logo ČVUT -- podle grafického manuálu ČVUT platného od prosince 2016. Pokud nevyhovuje PDF-verze, tak použijte jinou variantu loga: https://www.cvut.cz/logo-a-graficky-manual -> "Symbol a logo ČVUT v Praze"). Pokud chcete logo úplně vynechat, zadejte místo "\includegraphics{...}" text "\vspace{35mm}"

% přesně podle formuláře "Zadání bak./dipl. práce" VYPLŇTE:
\newcommand{\nazevczT}{Optické rozpoznávání znaků \\ na naskenovaných historických plakátech pomocí \\nejmodernějších metod}    % český název práce (přesně podle zadání!)
\newcommand{\nazevenT}{Optical Character Recognition \\on Scanned Historical Posters Using the State-of-the-Art Methods}          % anglický název práce (přesně podle zadání!)
\newcommand{\nazevcz}{Optické rozpoznávání znaků na naskenovaných historických plakátech pomocí nejmodernějších metod}    % český název práce (přesně podle zadání!)
\newcommand{\nazeven}{Optical Character Recognition on Scanned Historical Posters Using the State-of-the-Art Methods}          % anglický název práce (přesně podle zadání!)
\newcommand{\autor}{Anna Gruberová}   % vyplňte své jméno a příjmení (s akademickým titulem, máte-li jej)
\newcommand{\vedouci}{Ing. Adam Novozámský, Ph.D.} % vyplňte jméno a příjmení vedoucího práce, včetně titulů, např.: Doc. Ing. Ivo Malý, Ph.D.
\newcommand{\pracovisteVed}{Computer Vision Lab, Institute of Visual Computing \& Human-Centered Technology, TU Wien - Faculty of Informatics} % ZMĚŇTE, pokud vedoucí Vaší práce není z KSI
\newcommand{\konzultant}{--} % POKUD MÁTE určeného konzultanta, NAPIŠTE jeho jméno a příjmení
\newcommand{\pracovisteKonz}{--} % POKUD MÁTE konzultanta, NAPIŠTE jeho pracoviště

% podle skutečnosti VYPLŇTE:
\newcommand{\rok}{2022}  % rok odevzdání práce (jen rok odevzdání, nikoli celý akademický rok!)
\newcommand{\kde}{Praze} % studenti z Děčína ZMĚNÍ na: "Děčíně" (doplní se k "prohlášení")

\newcommand{\klicova}{.}   % zde NAPIŠTE česky max. 5 klíčových slov
\newcommand{\keyword}{.}       % zde NAPIŠTE anglicky max. 5 klíčových slov (přeložte z češtiny)
\newcommand{\abstrCZ}{.}


% zde NAPIŠTE abstrakt v češtině (cca 7 vět, min. 80 slov)
\newcommand{\abstrEN}{.} % zde NAPIŠTE abstrakt v angličtině

\newcommand{\prohlaseni}{Prohlašuji, že jsem svou bakalářskou práci vypracoval\woman{} samostatně a použil\woman{} jsem pouze podklady (literaturu, projekty, SW atd.) uvedené v přiloženém seznamu.} % text prohlášení můžete mírně upravit :-)

\newcommand{\podekovani}{.} % NAPIŠTE poděkování, např. svému vedoucímu:
% Děkuji Ing. Eleonoře Krtečkové, Ph.D. za vedení mé bakalářské práce a za podnětné návrhy, které ji obohatily.
% NEBO:
% Děkuji vedoucímu práce doc. Pafnutijovi Snědldítětikaši, Ph.D. za neocenitelné rady a pomoc při tvorbě bakalářské práce.

\newcommand{\ti}{\textit} % zkrácený příkaz pro kurzívu
\newcommand{\tb}{\textbf} % zkrácený příkaz pro tučné písmo
\newcommand{\tn}{\texttt} % zkrácený příkaz pro neproporcionalni písmo

% Vzhled kodu - lstlistings - python
\definecolor{codegreen}{rgb}{0,0.6,0}
\definecolor{codegray}{rgb}{0.5,0.5,0.5}
\definecolor{framegray}{rgb}{0.8,0.8,0.8}
\definecolor{codepurple}{rgb}{0.58,0,0.82}
\definecolor{backcolour}{rgb}{0.95,0.95,0.92}

\lstdefinestyle{mystyle}{
    language=Python,
    frame=single,
    rulecolor=\color{framegray},
    commentstyle=\color{codegreen},
    keywordstyle=\color{magenta}\bfseries,     numberstyle=\tiny\color{codegray},
    stringstyle=\color{codepurple},
    basicstyle=\ttfamily\footnotesize,
    breakatwhitespace=false,         
    breaklines=true,                 
    captionpos=t,                    
    keepspaces=true,                 
    numbers=left,                    
    numbersep=5pt,                  
    showspaces=false,                
    showstringspaces=false,
    showtabs=false,                  
    tabsize=2
}
\lstset{style=mystyle}

\renewcommand{\lstlistingname}{Function}

%-----------------------------------------------

\newenvironment{repl}
{\fontfamily{cmvtt}\selectfont \begin{mdframed}

}{\end{mdframed}}

\usepackage[linewidth=1pt]{mdframed}


\newcolumntype{L}{>{\centering\arraybackslash}m{2cm}}
\newcolumntype{N}{>{\centering\arraybackslash}m{1.7cm}}
\newcolumntype{M}{>{\centering\arraybackslash}m{1.6cm}}
\newcolumntype{K}{>{\centering\arraybackslash}m{1.5cm}}
\newcolumntype{S}{>{\centering\arraybackslash}m{1cm}}
\renewcommand{\arraystretch}{1.3}

\newcommand{\cmark}{\textcolor{green!80!black}{\ding{51}}}
\newcommand{\xmark}{\textcolor{red}{\ding{55}}}


