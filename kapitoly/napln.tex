
\begin{itemize}
    \item What is OCR
    \item Text detection
    \begin{itemize}
        \item CRAFT
    \end{itemize}
    \item Text recognition
    \item End-to-end systems
    \begin{itemize}
        \item[] Reading scanned documents
        \item EasyOCR
        \item keras-ocr
        \item Tesseract (PyTesseract)
        \item (Google Cloud Vision free) paid
        \item (AWS Recognition) paid 
        \item (Kili) paid
    \end{itemize}
    \item Results evaluation
    \begin{itemize}
        \item Comparison of output and ground-truth
        \item 
    \end{itemize}
    \item Testing methods on free datasets
    \item[] Description of datasets
    \item Using methods on historical posters    
    \item[] Description of dataset

\end{itemize}


\section{Scene text detection}

\section*{Methods}
\subsection{CRAFT}

\section{End-to-end systems}
\subsection{EasyOCR}
\subsection{Keras-ocr}

Keras-ocr is a python library used for detecting and recognizing text in images. It unites the CRAFT text detection model \footnote{hereinafter referred to as CRAFT} and an implementation in Keras python library of CRNN for recognizing text \footnote{hereinafter referred to as CRNN}. \cite{keras-ocr1}

CRNN already provides pretrained model which can be used directly without modification for recognition or it is used as initial model for training a new model on new data. 

\subsection{tesseract}





