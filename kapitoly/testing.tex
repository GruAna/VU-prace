\chapter{Testing methods}

I wrote all testing scripts and auxiliary functions in Python programming language. The scripts were written as Jupyter Notebook environment and saved in the corresponding file format \texttt{.ypnb}, lately these notebooks were saved as \texttt{.py} Python scripts. The set of utility functions is located in a file called \texttt{utils.py}.

In the experiments I chose three methods to be tested, namely, Tesseract, EasyOCR and keras-ocr. Each of them has corresponding Python package 

\section{EasyOCR}



\section{Tesseract}
Potreba cernobile a threshold tesseract to ma radsi, problem delat centralne - otsu a po castech ale stejne tezke u spousty obrazku s ruznym osvetlenim.
Tesseract zkousime u bitmap obr PSM 11 na Kaist a je o hodne horsi nez PSM 6, s 11 je to 9.6. proc. jinak u normlalnich obrazku je lepsi zase 11, taky treba o 9 procent rozdil.. pak jeste 4 to je obcas lepsi
U CTW datasetu je to s 6 horsi.

jeste hrani s barvama nebo upraveny , jak kde

verze
tesseract 4.0.0-beta.1
 leptonica-1.75.3
  libgif 5.1.4 : libjpeg 8d (libjpeg-turbo 1.5.2) : libpng 1.6.34 : libtiff 4.0.9 : zlib 1.2.11 : libwebp 0.6.1 : libopenjp2 2.3.0

 Found AVX2
 Found AVX
 Found SSE

\section{Keras}

Training took one hour and stopped after 57 epochs ,10.979148864746094,20.5778751373291. Continued training.